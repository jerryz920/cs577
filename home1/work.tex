%%%%%%%%%%%%%%%%%%%%%%%%%%%%%%%%%%%%%%%%%
% Short Sectioned Assignment
% LaTeX Template
% Version 1.0 (5/5/12)
%
% This template has been downloaded from:
% http://www.LaTeXTemplates.com
%
% Original author:
% Frits Wenneker (http://www.howtotex.com)
%
% License:
% CC BY-NC-SA 3.0 (http://creativecommons.org/licenses/by-nc-sa/3.0/)
%
%%%%%%%%%%%%%%%%%%%%%%%%%%%%%%%%%%%%%%%%%

%----------------------------------------------------------------------------------------
%	PACKAGES AND OTHER DOCUMENT CONFIGURATIONS
%----------------------------------------------------------------------------------------

\documentclass[paper=a4, fontsize=11pt]{scrartcl} % A4 paper and 11pt font size

\usepackage[T1]{fontenc} % Use 8-bit encoding that has 256 glyphs
\usepackage{fourier} % Use the Adobe Utopia font for the document - comment this line to return to the LaTeX default
\usepackage[english]{babel} % English language/hyphenation
\usepackage[fleqn]{amsmath}
\usepackage{amsfonts,amsthm} % Math packages
\usepackage{indentfirst}
\usepackage[top=0.8in, bottom=1in, left=0.75in, right=0.75in]{geometry}
\usepackage{listings}
\usepackage{courier} % Required for the courier font
\usepackage{inconsolata} % Required for inconsolata fonts
\usepackage{enumitem} % Used for the enumerate index
\usepackage{tikz} % Used for recursion tree

\usepackage{lipsum} % Used for inserting dummy 'Lorem ipsum' text into the template

\usepackage{sectsty} % Allows customizing section commands
\allsectionsfont{\normalfont\scshape} % Make all sections centered, the default font and small caps

\usepackage{fancyhdr} % Custom headers and footers
\pagestyle{fancy} % Makes all pages in the document conform to the custom headers and footers
\fancyhead{} % No page header - if you want one, create it in the same way as the footers below
\fancyfoot[L]{} % Empty left footer
\fancyfoot[R]{} % Empty center footer
\fancyfoot[C]{\thepage} % Page numbering for right footer
\renewcommand{\headrulewidth}{0pt} % Remove header underlines
\renewcommand{\footrulewidth}{0pt} % Remove footer underlines
\setlength{\headheight}{13.6pt} % Customize the height of the header
\setlength{\footskip}{20pt} % Customize the height of the header

\renewcommand*{\familydefault}{\ttdefault} %fonts setting
\usepackage[T1]{fontenc}
\newcommand{\Rn}[1]{\uppercase\expandafter{\romannumeral #1\relax}}

\numberwithin{equation}{section} % Number equations within sections (i.e. 1.1, 1.2, 2.1, 2.2 instead of 1, 2, 3, 4)
\numberwithin{figure}{section} % Number figures within sections (i.e. 1.1, 1.2, 2.1, 2.2 instead of 1, 2, 3, 4)
\numberwithin{table}{section} % Number tables within sections (i.e. 1.1, 1.2, 2.1, 2.2 instead of 1, 2, 3, 4)

%\setlength\parindent{0pt} % Removes all indentation from paragraphs - comment this line for an assignment with lots of text

%----------------------------------------------------------------------------------------
%	TITLE SECTION
%----------------------------------------------------------------------------------------

\newcommand{\horrule}[1]{\rule{\linewidth}{#1}} % Create horizontal rule command with 1 argument of height

\title{	
\normalfont \normalsize 
\textsc{University Wisconsin Madison, Computer Science Department} \\ [25pt] % Your university, school and/or department name(s)
\horrule{0.5pt} \\[0.4cm] % Thin top horizontal rule
\huge CS577 Assignment 1\\ % The assignment title
\horrule{2pt} \\[0.5cm] % Thick bottom horizontal rule
}

\author{Lei Kang, Liang Wang, Yan Zhai} % Your name

\date{\normalsize\today} % Today's date or a custom date

\begin{document}

\maketitle % Print the title

%----------------------------------------------------------------------------------------
%	PROBLEM 1
%----------------------------------------------------------------------------------------
\section*{Problem 1}

\begin{enumerate}[label={1.\arabic*}]
% Question 1.1
%------------------------------------------------
  \item The answer is $\Theta(NlogN)$\\
    Proof:
    \begin{itemize}
      \item $f(x)=log x$ is monotonic increase on $[1,\infty]$, so:
	\begin{flalign*}
	  \int^{n}_{n-1}log x\,dx \leq\,log n\,\leq \int^{n+1}_{n}log x\,dx
	\end{flalign*}

      \item using integration by parts formula, we have:
	\begin{flalign*}
	  \int^{n}_{1}log x\,dx = n log n + O(n log n)
	\end{flalign*}

      \item $f(n) \in O(n log n)$, because:
	\begin{flalign*}
	  f(n) = \sum_{i=1}^{n}log\,i 
	       \leq \sum_{i=2}^{n}\int^{i+1}_{i} log x\,dx
	       = \int^{n+1}_{2}log x\,dx
	       = O(n log n)
      	\end{flalign*}
      \item $f(n) \in \Omega(n log n)$, because:
	\begin{flalign*}
	  f(n) = \sum_{i=2}^{n}log\,i
	       \geq \sum_{i=2}^{n}\int^{i}_{i-1} log x\,dx
	       = \int^{n}_{1}log x\,dx
	       = \Omega(n log n)
      	\end{flalign*}
      \item combined above points, so $f(n) \in \Theta(n log n)$
    \end{itemize}

% Question 1.2
%------------------------------------------------
  \item use Stirling approximation and logarithm property, we have:
    \begin{flalign*}
      f_8(n) < f_4(n) < f_3(n) < f_7(n) < f_1(n) < f_5(n) < f_2(n) < f_6(n)
    \end{flalign*}

\end{enumerate}


%----------------------------------------------------------------------------------------
%	PROBLEM 2
%----------------------------------------------------------------------------------------
\section*{Problem 2}

\begin{enumerate}[label={2.(\alph*)}]
% Question 2.a
%------------------------------------------------
  \item We have following definitions:
    \begin{align*}
      & T_1(n) = O(n) + T_1(\frac{3n}{4}) + T_1(\frac{n}{4}) & (\Rn{1}) \\
      & T_2(n) = 2T(n-1) + O(1) & (\Rn{2}) \\
      & T_3(n) = T((1-\epsilon)n) + O(n^2) & (\Rn{3})  \\
      & T_4(n) = \sqrt{n}T(\sqrt{n}) + O(n) & (\Rn{4}) 
    \end{align*}

% Question 2.b
%------------------------------------------------
  \item 
    \begin{enumerate}[label={\roman*.}]
      \item Using recursion tree method to expand the expression, we could know
	that at each level of the tree, it would require $O(n)$ time to
	process. Although the tree is not a full binary tree, we could infer its size
	should be between a full binary tree with height $log_4n$ and
	$log_\frac{4}{3}n$. So we have following inequality:
	\begin{align*}
	  O(n) \times log_4n \leq T_1(n) \leq O(n) \times log_\frac{4}{3}n
	\end{align*}
	The lower bound and upper bound differs only in constant coefficient, thus:
	\begin{align*}
	  T_1(n)=O(n log n)
	\end{align*}

      \item This simple equation could be recursively expanded as:
	\begin{align*}
	  T_2(n) = \sum_{i=2}^{n} O(2^i) = 2^{(n+1)}-2 = O(2^n)
	\end{align*}

      \item According to master theorem, since the coefficient of subproblem is 1,
	we could directly infer that:
	\begin{align*}
	  T_3(n) = O(n^2)
	\end{align*}

      \item Again use the recursion tree to expand the $T_4(n)$, we will get a
	full tree of height $log_2log_2n$, and at each level, we need $\sqrt{n}
	\times O(\sqrt{n}) = O(n)$ to process. So we could easily calculate:
	\begin{align*}
	  T_4(n) = log_2log_2n \times O(n) = O(n log_2log_2n)
	\end{align*}

    \end{enumerate}

% Question 2.c
%------------------------------------------------
  \item $T_4(n) < T_1(n) < T_3(n) < T_4(n)$

\end{enumerate}


%----------------------------------------------------------------------------------------
%	PROBLEM 3
%----------------------------------------------------------------------------------------
\section*{Problem 3}
%------------------------------------------------
\begin{enumerate}[label={3.\arabic*}]
% Question 3.1
%------------------------------------------------
  \item when $b=0$, function returns at first 'if' branch, so it means $a^0=1$, which is apparently true.

% Question 3.2
%------------------------------------------------
  \item this inductive base is not as expressive as its alternative form: for $b \leq b^*$, the algorithm works. The proof contains two part:
    \begin{itemize}
      \item When $b+1$ is an odd number, \textit{fastExp}$(a,b+1)=a\times$\textit{fastExp}$(a,b)$. Use the inductive base, we have got \textit{fastExp}$(a,b+1) = a^{b+1}$
      \item When $b+1$ is an even number, \textit{fastExp}$(a,b+1)=$\textit{fastExp}$(a, \frac{b}{2})^2$ 
    \end{itemize}
    Now using the induction, we have concluded \textit{fastExp} works for any $n \in \mathbb{N}$

\end{enumerate}

%----------------------------------------------------------------------------------------
%	PROBLEM 4
%----------------------------------------------------------------------------------------
\section*{Problem 4}
%----------------------------------------------------------------------------------------
\begin{enumerate}[label={4.\arabic*}]
% Question 4.1
%------------------------------------------------
  \item one possible solution is:
    \begin{flalign*}
      & f(n) = (N-1)!, \;\;n \in [(N-1)!, N!),\;\;N \in \mathbb{N}\\
      & g(n) = \frac{n}{2}
    \end{flalign*}
    It's not hard to see that $g(N!) < f(N!)$ and $g(3N!) > f(3N!)$ when $N$ is large enough, thus
    $f(n) \notin O(g(n))$ and $g(n) \notin O(f(n))$.

% Question 4.2
%------------------------------------------------
  \item it suffices to show that for some pair of functions, such $h$ does not exist.\\
    Let $f_1,f_2$ be $f,g$ we defined above. If there is a $h$ so that 
    \begin{flalign*}
      & f_1(n) \in O(h(n)) \\
      & f_2(n) \in O(h(n)) \\
      & h(n) \in O(f_i(n)), i \in {1,2}
    \end{flalign*}
    assume $i$ is $1$, then $f_2(n) \in O(h(n)) \in O(O(f_1(n))) \in O(f_1(n))$. But this
    contradicts the fact $f(n) \notin O(g(n))$ and $g(n) \notin O(f(n))$.

% Question 4.3
%------------------------------------------------
  \item for any finite set of such functions, let $h(n)= \sum^{k}_{i=1}f_i(n)$, we have:
    \begin{flalign*}
      f_i(n) \in O(f_i(n)) \in O(\sum&{k}_{i=1}f_i(n)) \in O(h(n)) \\
      f_i(n) \in O(h'(n)) & \rightarrow h(n) \in \sum^{k}_{i=1} O(h'(n)) \\
			  & \rightarrow h(n) \in \O(kh'(n))\\
     			  & \xrightarrow{\text{$k$ is finite constant}} h(n) \in \O(h'(n))\\
    \end{flalign*}
    this means for any $h'$ satisfies condition(1), we have $h(n) \in O(h'(n))$.

\end{enumerate}

\end{document}
